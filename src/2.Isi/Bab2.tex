%-----------------------------------------------------------------------------%
\chapter{\babDua}
%-----------------------------------------------------------------------------%

%-----------------------------------------------------------------------------%
\section{Sejarah Singkat Perusahaan}
%-----------------------------------------------------------------------------%
PT Visi Karya Nusantara didirikan pada tahun 2025, berdasarkan informasi yang diperoleh melalui wawancara langsung dengan Bapak Raditya selaku \textit{Software Engineer Manager}. Sebagai perusahaan konsultan teknologi informasi, PT Visi Karya Nusantara berfokus pada penyediaan solusi IT yang inovatif dan efektif bagi berbagai klien di Indonesia. Dalam kurun waktu yang singkat, perusahaan telah berhasil membangun reputasi yang solid di industri ini dengan menyelesaikan berbagai proyek menantang dan memberikan nilai tambah yang signifikan. Berlokasi di Start Space Coworking Space Gading Serpong, Tangerang, PT Visi Karya Nusantara saat ini aktif mengerjakan beberapa proyek utama, di antaranya adalah Nine to Six (NTS), AKS, dan KAIYA \cite{transkrip2025}. Logo resmi perusahaan dapat dilihat pada Gambar \ref{fig:logo_perusahaan}.
\begin{figure}
	\centering
	\fbox{\includegraphics[width=0.4\textwidth]{assets/pics/fig_logo_perusahaan.png}}
	\caption{Logo Perusahaan PT Visi Karya Nusantara}
	\vspace{0em}
	% \vspace{-1em} %kalau 2 baris
	% {\small Sumber: \cite{Widjaja2002a}}
	\label{fig:logo_perusahaan}
\end{figure}
Gambar \ref{fig:logo_perusahaan} merupakan logo dari perusahaan PT Visi Karya Nusantara. Berdasarkan wawancara langsung dengan Bapak Raditya selaku \textit{Software Engineer Manager}, 

%-----------------------------------------------------------------------------%
\section{Visi dan Misi Perusahaan}
%-----------------------------------------------------------------------------%
Visi dan misi perusahaan PT Visi Karya Nusantara diperoleh melalui wawancara langsung dengan Bapak Raditya selaku \textit{Software Engineer Manager}. 

Visi dari PT Visi Karya Nusantara adalah:  
\begin{quote}
    \textit{"Menjadi perusahaan konsultan teknologi informasi yang memberikan solusi yang optimal, efektif, dan efisien, serta membangun budaya perusahaan yang berintegritas, profesional, dan berkelanjutan."}
\end{quote}

Misi dari PT Visi Karya Nusantara adalah:
\begin{enumerate}
    \item Menyediakan layanan konsultasi teknologi informasi yang berorientasi pada efektivitas dan efisiensi, guna memastikan setiap solusi yang diberikan selaras dengan kebutuhan dan tujuan klien.
    \item Membangun organisasi yang profesional dan berintegritas melalui pengembangan sumber daya manusia yang adaptif, kompeten, dan berdaya saing, serta menumbuhkan budaya kerja yang kolaboratif, inovatif, dan berkelanjutan.
    \item Berperan aktif dalam mendukung transformasi digital di Indonesia dengan mengedepankan praktik konsultasi yang beretika, berkualitas, dan berorientasi pada hasil jangka panjang.
\end{enumerate}

%-----------------------------------------------------------------------------%
\section{Struktur Organisasi Perusahaan}
%-----------------------------------------------------------------------------%
% Hal pertama yang mungkin ditanyakan adalah bagaimana membuat huruf tercetak 
% tebal, miring, atau memiliki garis bawah. 
% Pada Texmaker, Anda bisa melakukan hal ini seperti halnya saat mengubah dokumen 
% dengan OO Writer. 
% Namun jika tetap masih tertarik dengan cara lain, ini dia: 

% \begin{itemize}
% 	\item \bo{Bold} \\
% 		Gunakan perintah \bslash$\lbrace\rbrace$ atau 
% 		\bslash bo$\lbrace\rbrace$. 
% 	\item \f{Italic} \\
% 		Gunakan perintah \bslash textit$\lbrace\rbrace$ atau 
% 		\bslash f$\lbrace\rbrace$. 
% 	\item \underline{Underline} \\
% 		Gunakan perintah \bslash underline$\lbrace\rbrace$.
% 	\item $\overline{Overline}$ \\
% 		Gunakan perintah \bslash overline. 
% 	\item $^{superscript}$ \\
% 		Gunakan perintah \bslash $\lbrace\rbrace$. 
% 	\item $_{subscript}$ \\
% 		Gunakan perintah \bslash \_$\lbrace\rbrace$. 
% \end{itemize}

% Perintah \bslash f dan \bslash bo hanya dapat digunakan jika package 
% uithesis digunakan untuk panduan. 
% Struktur organisasi perusahaan PT Visi Karya Nusantara diperoleh dari dokumen internal resmi perusahaan berjudul \textit{New Employee Onboarding} \cite{newemployeeonboarding2025}.
Struktur organisasi perusahaan PT Visi Karya Nusantara dapat dilihat pada Gambar \ref{fig:struktur_organisasi_perusahaan}.
\begin{figure}
	\centering
	\fbox{\includegraphics[width=0.8\textwidth]{assets/pics/fig_struktur_organisasi_perusahaan.jpg}}
	\caption{Struktur organisasi perusahaan PT Visi Karya Nusantara}
	\vspace{0em}
	% \vspace{-1em} %kalau 2 baris
	% {\small Sumber: \cite{Widjaja2002a}}
	\label{fig:struktur_organisasi_perusahaan}
\end{figure}

Struktur organisasi PT Visi Karya Nusantara terdiri atas tiga tingkatan utama yang membentuk alur koordinasi kerja secara efektif dan terarah. Pada tingkat tertinggi, terdapat Director, yang memegang tanggung jawab utama dalam menentukan visi, arah strategis, serta pengambilan keputusan penting perusahaan.

Di bawahnya terdapat Software Engineer Manager, yang berperan dalam mengatur dan mengawasi seluruh kegiatan teknis yang berkaitan dengan pengembangan perangkat lunak. Posisi ini juga bertugas untuk memberikan arahan teknis, melakukan evaluasi hasil kerja tim, serta memastikan proyek pengembangan berjalan sesuai standar dan tenggat waktu yang telah ditetapkan.

Selanjutnya, di bawah koordinasi \textit{Software Engineer Manager}, terdapat dua peran penting yaitu Software Engineer dan UI/UX Designer. \textit{Software Engineer} bertanggung jawab atas proses pengembangan aplikasi, mulai dari implementasi fitur hingga integrasi sistem. Sementara itu, \textit{UI/UX Designer} berfokus pada perancangan antarmuka pengguna serta pengalaman pengguna agar produk yang dikembangkan tidak hanya berfungsi dengan baik, tetapi juga memiliki tampilan yang menarik dan mudah digunakan.

Struktur organisasi ini menggambarkan alur kerja yang ringkas namun efektif, dengan fokus pada kolaborasi lintas peran untuk menghasilkan produk digital yang berkualitas tinggi dan sesuai dengan kebutuhan pengguna.

% %-----------------------------------------------------------------------------%
% \section{Memasukan Gambar}
% %-----------------------------------------------------------------------------%
% Setiap gambar dapat diberikan caption dan diberikan label. Label dapat 
% digunakan untuk menunjuk gambar tertentu. 
% Jika posisi gambar berubah, maka nomor gambar juga akan diubah secara 
% otomatis. 
% Begitu juga dengan seluruh referensi yang menunjuk pada gambar tersebut. 
% Contoh sederhana adalah \pic~\ref{fig:testGambar}. 
% Silahkan lihat code \latex~dengan nama bab2.tex untuk melihat kode lengkapnya. 
% Harap diingat bahwa caption untuk gambar selalu terletak dibawah gambar. 