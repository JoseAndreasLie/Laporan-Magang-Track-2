%-----------------------------------------------------------------------------%
\chapter*{\Judul}
%-----------------------------------------------------------------------------%
\singlespacing
\begin{center}
    
    \vspace{-4em}
    
    \penulis
    
	\bigskip
    
    \textbf{ABSTRAK}
    
\end{center}

% \chapter*{Abstrak}

\vspace*{0.2cm}
{
	\setlength{\parindent}{0pt}

	\bigskip
	\bigskip

	Laporan ini menjelaskan pengembangan sistem informasi kepegawaian Nine to Six (NTS) di PT Visi Karya Nusantara, sebuah perusahaan konsultan teknologi informasi yang berfokus pada solusi digital inovatif. Digitalisasi pengelolaan sumber daya manusia (SDM) telah menjadi kebutuhan kritis untuk meningkatkan efisiensi operasional, akurasi penggajian, dan kepatuhan regulasi. Penelitian menunjukkan bahwa integrasi modul-modul seperti \textit{Contract Management}, \textit{Overtime Management}, dan \textit{Payslip} secara signifikan mengurangi kesalahan manual dan meningkatkan transparansi data. Proyek ini dirancang untuk mengembangkan sistem terpadu yang menghubungkan data statis (kontrak karyawan), data transaksional (kehadiran dan lembur), dan proses finalisasi kompensasi (\textit{payslip}) dalam satu platform berbasis \textit{cloud}. Menggunakan PERN \textit{Stack} (PostgreSQL, Express.js, React.js, Node.js), sistem dikembangkan dengan metodologi \textit{Agile} selama periode magang dari 4 Agustus 2025 hingga 4 Januari 2026. Hasil pengembangan mencakup implementasi modul \textit{Contract Management}, \textit{Payroll} dengan komponen otomatis, \textit{Overtime Management}, \textit{On Boarding} berbasis CSV, serta \textit{Payslip} terintegrasional dengan sistem \textit{Daily Attendance}. Sistem ini diharapkan meningkatkan efisiensi HR, menjamin akurasi perhitungan kompensasi, dan memberikan transparansi penuh kepada karyawan	 melalui akses \textit{self-service}.

	\bigskip
 
% Kata kunci urut abjad
% 3 – 5 kata kunci
	\textbf{Kata Kunci}: \textit{Backend development, Express, Human resource information system, On boarding, Payroll.}	
}

\onehalfspacing