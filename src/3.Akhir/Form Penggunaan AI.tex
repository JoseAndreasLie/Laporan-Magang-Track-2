\noindent
\begin{tabular}{lcl}
   Nama  &:& \penulis\\
   NIM  &:& \nim\\
   Email &:& jose.andreas1@student.umn.ac.id \\
   Program Studi &:& \program\\
%   Fakultas &:& \fakultas
\end{tabular}

\noindent
% \usepackage{array}
\setlength{\LTleft}{0pt}
\setlength{\LTright}{0pt}
\setlength{\extrarowheight}{2pt}

{\footnotesize
\begin{longtable}{
|p{0.4cm}
|p{2cm}
|p{3cm}
|m{4cm}
|p{1.5cm}
|m{3cm}|}   % m{} for image
\hline

\centering\textbf{No} &
\centering\textbf{Nama Tool} &
\centering\textbf{Alamat Web/Url} &
\centering\textbf{Prompt} &
\centering\textbf{Tanggal Akses} &
\centering\textbf{Media Output}
\hline
\endfirsthead

\hline
\centering\textbf{No} &
\centering\textbf{Nama Tool} &
\centering\textbf{Alamat Web/Url} &
\centering\textbf{Prompt} &
\centering\textbf{Tanggal Akses} &
\centering\textbf{Media Output}
\hline
\endhead

\centering 1 &
\centering Gemini &
\centering gemini.google.com &
Dalam sistem HRIS modern, jelaskan perbedaan antara data statis dan data transaksional. \ldots&
\centering 01 Agustus 2025 &
\centering\includegraphics[width=\linewidth]{assets/pics/prompt-1.png} \hline

\centering 2 &
\centering Gemini &
\centering gemini.google.com &
Analisis pengaruh payroll automation dalam meningkatkan kepatuhan terhadap regulasi \ldots&
\centering 01 Agustus 2025 &
\centering\includegraphics[width=\linewidth]{assets/pics/prompt-2.png} \hline

\centering 3 &
\centering Gemini &
\centering gemini.google.com &
Jelaskan peran kritis Modul Kontrak Karyawan sebagai data master dalam HRIS \ldots&
\centering 05 Agustus 2025 &
\centering\includegraphics[width=\linewidth]{assets/pics/prompt-3.png} \hline

\centering 4 &
\centering Gemini &
\centering gemini.google.com &
Jelaskan dengan detail konsep "single source of truth" dalam HRIS \ldots&
\centering 05 Agustus 2025 &
\centering\includegraphics[width=\linewidth]{assets/pics/prompt-4.png} \hline

\end{longtable}
}



